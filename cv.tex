% LaTeX using res.cls
\documentclass[line,11pt]{res}
\title{CV}
%\usepackage{helvetica} % uses helvetica postscript font (download helvetica.sty)
%\usepackage{newcent}   % uses new century schoolbook postscript font  
\setlength{\topmargin}{-0.6in}  % Start text higher on the page 
\setlength{\textheight}{9.8in}  % increase textheight to fit more on a page
\setlength{\headsep}{0.2in}     % space between header and text
\setlength{\headheight}{13.6pt}   % make room for header
\usepackage{fancyhdr}  % use fancyhdr package to get 2-line header
\renewcommand{\headrulewidth}{0pt} % suppress line drawn by default by fancyhdr
\lhead{\hspace*{-\sectionwidth}Daniel N. Hill} % force lhead all the way left
\rhead{Page \thepage}  % put page number at right
\cfoot{}  % the footer is empty
\pagestyle{fancy} % set pagestyle for the document

\begin{document} 
\thispagestyle{empty} % this page does not have a header
\name{Daniel N. Hill}
\address{246 $2^{nd}$ St. \#1401, San Francisco, CA 94105}
\address{716.771.8224}
\address{daniel.n.hill@gmail.com}
\address{www.danielnhill.com}

\begin{resume}

\section{Research Positions}
\vspace{0.1in} 

 {\bf Machine Learning Scientist}, 2015-present \\ Amazon.com, Inc., Palo Alto, CA 
  \begin{itemize}
          \item[] Developed ranking algorithms for Amazon Interesting Finds using multi-armed bandit methods, diversification, and personalization.  Collaborated with Prof. SVN Vishwanathan of UCSC.  
         \item[] Created a multivariate testing system in which experiments with over 100 treatments converged 10 times faster than traditional A/B testing.  
   \end{itemize}
   
 {\bf Senior Data Scientist}, 2013-2015 \\ Integral Ad Science, New York, NY 
  \begin{itemize}
          \item[] Led causal analysis project to estimate the return on investment of digital ad campaigns using observational analysis when A/B tests are unavailable.  Collaborated with Prof. Foster Provost of NYU and Prof. Alan Hubbard of Berkeley.  
   \end{itemize}
   
 {\bf Post-doctoral Researcher}, 2010-2012 \\ Technical University of Munich, Germany 
 \\ Advisor: Prof. Arthur Konnerth
  \begin{itemize}
        \item[] Performed high-speed calcium imaging in vivo to understand information processing within the dendrites of L5 cortical neurons.  Collaborated with Nobel laureate Bert Sakmann.
  \end{itemize}
    
{\bf PhD Student}, 2003-2009 \\ UCSD Neurophysics Lab, San Diego, CA \\ Advisor: Prof. David Kleinfeld
  \begin{itemize}
        \item[] Recorded and analyzed multi-channel electrophysiological and videographic data sets to investigate control of the whisker system by primary motor cortex.
         \item[] Built a biomechanical simulator of the whisking apparatus based on my data.
        \item[] Developed a MATLAB toolbox called UltraMegaSort2000 which is used for analyzing electrophysiological data by dozens of research labs worldwide.
 \end{itemize}    
    
   {\bf Research Assistant}, 2001-2002 \\ Los Alamos National Laboratory, New Mexico  
   \\ Advisor: Dr. Garrett Kenyon
  \begin{itemize}
        \item[] Coded neural network model of retina to test hypothesis about Benham's Top illusion.
  \end{itemize}

   {\bf Research Assistant}, 2001 \\ Los Alamos National Laboratory, New Mexico  
   \\ Advisor: Dr. Bill Hlavacek
  \begin{itemize}
  \item[] Implemented and parallelized clustering algorithms that I applied to DNA sequences, genome expression patterns, and lightning classification. 
  \end{itemize}

\section{Education}
\vspace{0.1in} 
 
 	\textbf{Computational Neuroscience, PhD}, 2009 \\
    University of California, San Diego
 
 	\textbf{Electrical and Computer Engineering, MS}, 2008 \\    
    Specialization in Intelligent Systems, Robotics, and Control.\\
    Univiersity of California, San Diego
    
 	\textbf{Summer course in Neuroinformatics}, 2006 \\
    Marine Biology Laboratory, Woods Hole, MA 
    
 	\textbf{Computer Science, BS}, 2002 \\
    Concentration in Artificial Intelligence. \\
    Rochester Institute of Technology
    
\section{Fellowships}
\vspace{0.1in} 
 
 	\textbf{NRSA}, National Institutes of Health, 2006-2008
    
    \textbf{IGERT}, National Science Foundation, 2002-2005

\section{Patents}
\vspace{0.1in}
\textbf{Hill DN}, Tsemekhman K. Methods, systems, and media for identifying automatically refreshed advertisements. U.S. Patent Application 14/329,514, filed August 2014. Patent Pending.

\section{Participation in Scientific Community}
\vspace{0.1in} 

	\textbf{Neural Information Processing Systems Workflow}, Workflow Chair 2017 \\
    	Organized review process for 3,300 submissions and 2,200 reviewers.
    
	\textbf{AMC Summer School on Data Science and Big Data}, Contributor 2017 \\
         Developed class materials for full-day course on recommender systems
         
	\textbf{Neuroinformatics Summer Course}, Lecturer 2008-2012 \\
    Marine Biology Laboratory, Woods Hole, MA
    
    	\textbf{Chronux Project}, Scientific Software Contributor 2006-2009 \\
    	Open-source, community-driven, NIH-sponsored software project. I developed software to perform spike sorting, a signal detection and data clustering problem encountered in systems neuroscience. 
    
\section{Papers}
\vspace{0.1in}
\textbf{Hill DN}, Nassif H, Liu Y, Iyer A, Vishwanathan SVN. An efficient bandit algorithm for realtime multivariate optimization. In Proceedings of the 23rd ACM SIGKDD International Conference on Knowledge Discovery and Data Mining. (2017)

Teo CH, Nassif H, \textbf{Hill D}, Srinivasan S, Goodman M, Mohan V, Vishwanathan SVN. Adaptive, Personalized Diversity for Visual Discovery. In Proceedings of the 10th ACM Conference on Recommender Systems, pp. 35-38. (2016) \textbf{Oral presentation. Winner of best paper award.}

\textbf{Hill DN}, Moakler R, Hubbard AE, Tsemekhman V, Provost F, Tsemekhman K.  Measuring causal impact of online actions via natural experiments: application to display advertising. In Proceedings of the 21th ACM SIGKDD International Conference on Knowledge Discovery and Data Mining, pp. 1839-1847. (2015) \textbf{Oral presentation.}

\textbf{Hill DN}, Varga Z, Jia H, Sakmann B, Konnerth A. Multibranch activity in basal and tuft dendrites during firing of layer 5 cortical neurons in vivo. PNAS. 110(33):13618-13623 (2013)

Grienberger C, Rochefort NL, Adelsberger H, Henning HA, \textbf{Hill DN}, Reichwald J, Staufenbiel M, Konnerth A. Staged decline of neuronal function in vivo in an animal model of Alzheimer's disease. Nat Commun. 3:774. (2012) 

\textbf{Hill DN}, Curtis J, Moore JD, Kleinfeld D.  Primary motor cortex reports efferent control of vibrissa position on multiple timescales. Neuron. 72(2):344-56 (2011).

Rochefort NL, Narushima M, Grienberger C, Marandi N, \textbf{Hill DN}, Konnerth A. Development of direction selectivity in mouse cortical neurons. Neuron. 71(3):425-32 (2011). 

\textbf{Hill DN}, Mehta SB, Kleinfeld D.  Quality metrics to accompany spike sorting of extracellular signals. J. Neurosci. 31:8699-8705 (2011)

Wolfe J, \textbf{Hill DN}, Pahlavan S, Drew PJ, Kleinfeld D, Feldman DE. Texture coding in the rat whisker system: slip-stick versus differential encoding. PLoS Biology. 6(8):1661-1677 (2008). 

\textbf{Hill DN}, Bermejo R, Zeigler HP, Kleinfeld D. Biomechanics of the vibrissa motor plant in rat: Rhythmic whisking consists of triphasic neuromuscular activity. J Neurosci. 28(13):3438-55 (2008). 

Kenyon GT, \textbf{Hill D}, Theiler J, George JS, Marshak DW. A theory of the Benham top based on center-surround interactions in the parvocellular pathway. Neural Networks. 17: 773-786 (2004). 

Eads D, \textbf{Hill D}, Davis S, Perkins S, Ma J, Porter R, Theiler J. Zeus: Genetic algorithms and support vector machines for time series classification. Proc. SPIE. 4787:74-85 (2002).

\section{Book Chapter}
\vspace{0.1in}

\textbf{Hill DN}, Kleinfeld D, Mehta SB. Spike Sorting. In Observed Brain Dynamics by P. P.Mitra and H. Bokil, Oxford Press. (2007)

\section{Invited Talks}
\vspace{0.1in}

Adaptive, Personalized Diversity for Visual Discovery. Netflix workshop on Personalization, Recommendation, and Search. Los Altos, CA (2017)

Adaptive, Personalized Diversity for Visual Discovery. BayLearn, Sunnyvale, CA (2017)

Using negative controls to validate and repair a natural online experiment.  Conference on Digital Experimentation at MIT, Boston, MA (2014).

How single are my units? Quality metrics in spike sorting. Validation of Automatic Spike-Sorting Methods workshop, Ski, Norway (2011)

The cortical representation of muscle activation in the control of whisking. Barrels XXI, Baltimore, MD (2008)

Triphasic neuromuscular control of whisking. Functional Organization of Barrel Cortex Networks. Alicante, Spain (2008)

Biomechanics, behavior, and encoding in the rat vibrissa system. COSYNE, Park City, UT (2007)

Biomechanics of whisking: whisking consists of 3 phases of muscle activity. Barrels XIX, Atlanta, GA (2006)

Electromyography and kinematics of whisking: a preliminary analysis. Barrels XVII, San Diego, CA (2004)

\end{resume}

\end{document}

