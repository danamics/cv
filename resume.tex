% LaTeX resume using res.cls
\documentclass[line,10pt]{res}
\title{Resume}
%\usepackage{helvetica} % uses helvetica postscript font (download helvetica.sty)
%\usepackage{newcent}   % uses new century schoolbook postscript font  
\setlength{\topmargin}{-0.6in}  % Start text higher on the page 
\setlength{\textheight}{9.9in}  % increase textheight to fit more on a page
\setlength{\headsep}{0.13in}     % space between header and text
\setlength{\headheight}{13.6pt}   % make room for header
\usepackage{fancyhdr}  % use fancyhdr package to get 2-line header
\renewcommand{\headrulewidth}{0pt} % suppress line drawn by default by fancyhdr
\lhead{\hspace*{-\sectionwidth}Daniel N. Hill} % force lhead all the way left
\rhead{Page \thepage}  % put page number at right
\cfoot{}  % the footer is empty
\pagestyle{fancy} % set pagestyle for the document

\begin{document} 
\thispagestyle{empty} % this page does not have a header
\name{Daniel N. Hill}
\address{716.771.8224}
\address{daniel.n.hill@gmail.com}

\begin{resume}

\section{EDUCATION}
 \textbf{Computational Neuroscience, PhD}, 2009, University of California, San Diego \\
 \textbf{Electrical and Computer Engineering, MS}, 2008, University of California, San Diego \\
 \textbf{Computer Science, BS}, 2002, Rochester Institute of Technology
\section{EXPERIENCE}
 {\bf Senior Applied Scientist}, 2018-present \\ Amazon.com, Search, Berkeley, CA 
  \begin{itemize}
          \item[] I manage a machine learning team focused on bandit methods and session-awareness for autocomplete and product search.  We productionized Deep Learning and Extreme Multilabel Ranking models.
   \end{itemize}

 {\bf Applied Scientist}, 2015-2018 \\ Amazon.com, Personalization, Palo Alto, CA 
  \begin{itemize}
          \item[] Developed ranking algorithms using multi-armed bandit methods, diversification, and personalization. Created the engine for a multivariate testing system in which experiments with over 100 treatments converge 10 times faster than traditional A/B testing.  
   \end{itemize}
   
 {\bf Senior Data Scientist}, 2013-2015 \\ {\bf Integral Ad Science}, New York, NY 
  \begin{itemize}
          \item[] {\small Led ``Causal Impact'' project to estimate ROI of digital ad campaigns using observational analysis when A/B tests are unavailable.  }
        \end{itemize}
{\bf Post-doctoral Researcher}, 2010-2012 \\ {\bf Technical University of Munich}, Germany 
  \begin{itemize}
        \item[] {\small 
Recorded and analyzed high frame rate video of calcium activity in neuronal dendrites.  Worked in collaboration with Nobel Laureate Bert Sakmann.        }
  \end{itemize}

\section{SKILLS} 
{\bf Programming:} {\small Python, PyTorch, Java, Spark, Git, \LaTeX, bash, docker, HTML, AWS} \\
{\bf Statistics and Machine Learning: } {\small Multi-armed bandits, recommender systems, NLP, XMC, RNN}
    
\section{SELECTED PUBLICATIONS} 

Sen R, Rakhlin A, Ying L, Kidambi R, Fostesr D, \textbf{Hill DN}, Dhillon IS. Top-k eXtreme Contextual Bandits with Arm Hierarchy. Submitted. (2021)

Yadav N, Sen R, \textbf{Hill DN}, Mazumdar A, Dhillon IS. Session-Aware Query Auto-completion using Extreme Multi-label Ranking. Submitted. (2021)

Ai Q, \textbf{Hill DN}, Vishwanathan SVN, Croft WB. A Zero Attention Model for Personalized Product Search. In Proceedings of the 28th ACM International Conference on Information and Knowledge Management. (2019)

\textbf{Hill DN}, Nassif H, Liu Y, Iyer A, Vishwanathan SVN. An efficient bandit algorithm for realtime multivariate optimization. In Proceedings of the 23rd ACM SIGKDD International Conference on Knowledge Discovery and Data Mining. (2017) \textbf{Winner of Audience Appreciation Award.}

Teo CH, Nassif H, \textbf{Hill D}, Srinivasan S, Goodman M, Mohan V, Vishwanathan SVN. Adaptive, Personalized Diversity for Visual Discovery. In Proceedings of the 10th ACM Conference on Recommender Systems, pp. 35-38. (2016) \textbf{Oral presentation. Winner of best paper award.}

\textbf{Hill DN}, Moakler R, Hubbard AE, Tsemekhman V, Provost F, Tsemekhman K.  Measuring causal impact of online actions via natural experiments: application to display advertising. In Proceedings of the 21th ACM SIGKDD International Conference on Knowledge Discovery and Data Mining, pp. 1839-1847. (2015) \textbf{Oral presentation.}

\end{resume}
\end{document}






























